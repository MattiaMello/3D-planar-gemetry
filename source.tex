\documentclass[12pt]{article}
\usepackage[margin=1.5cm]{geometry}
\usepackage[inline]{asymptote}
\usepackage{graphicx}
\usepackage{subcaption} % https://latex-tutorial.com/tutorials/figures/

\begin{document}
	
	this is a sample of 3D generated imagery using LaTeX

\begin{figure}[h!]
    \centering
    \begin{asy}[width=10cm]
    size(300);
    
    void testpolygon(pair[] A, int n=3, real s=0.2, pen p1=yellow, pen p2=blue)
    {
    pair[] a=copy(A);
    guide g= operator --(... a)--cycle;
    guide pic()
    {
    pair M=a[0];
    for (int i=0; i<a.length; ++i)
    {
    if (i != (a.length-1))
    {
    a[i]=relpoint(a[i]--a[i+1],s);
    } else {
    a[i]=relpoint(a[i]--M,s);
    }
    }
    return operator --(... a)--cycle;
    }
    for (int i=0; i <= n; ++i)
    {
    if (i==0) filldraw(g,p1);
    else filldraw(pic(),(i%2==1) ? p2 : p1);
    }
    }
    
    int n=25;
    path g=rotate(-60)*polygon(6);
    testpolygon(new pair[]{(0,0),point(g,0),point(g,1)},n,0.1);
    testpolygon(new pair[]{(0,0),point(g,1),point(g,2)},n,0.1);
    testpolygon(new pair[]{(0,0),point(g,2),point(g,3)},n,0.1);
    testpolygon(new pair[]{(0,0),point(g,3),point(g,4)},n,0.1);
    testpolygon(new pair[]{(0,0),point(g,4),point(g,5)},n,0.1);
    testpolygon(new pair[]{(0,0),point(g,5),point(g,0)},n,0.1);
    \end{asy}
\end{figure}
\end{document}